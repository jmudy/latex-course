
\documentclass[12pt]{article}
\usepackage[spanish]{babel}
\usepackage{url}

\title{Diez secretos para dar una buena charla científica}
\author{Jesús Mudarra Luján}
\date{\today}


\begin{document}
\maketitle

\section{Introducción}

El texto de este ejercicio es una versión significativamente abreviada y ligeramente modificada del excelente artículo del mismo nombre de Mark Schoeberl y Brian Toon: \url {http://www.cgd.ucar.edu/cms/agu/scientific_talk.html}

\section{Los secretos}

He compilado esta lista personal de ``secretos'' tras escuchar a oradores efectivos y a oradores ineficaces. No pretendo que esta lista sea exhaustiva, estoy seguro de que hay cosas que me he olvidado. Pero mi lista probablemente cubra alrededor del 90\% de lo que necesita saber y hacer.

\begin{enumerate}

\item Prepara tu material con cuidado y con lógica. Cuenta una historia.

\item Practica tu charla. No hay excusas para la falta de preparación.

\item No incluyas demasiado material. Los buenos oradores tienen uno o dos puntos centrales y se apegan a ellos.

\item Evita las ecuaciones. Se dice que por cada ecuación en una charla, la cantidad de personas que la entenderán se reducirá a la mitad. Es decir, si $q$ es el número de ecuaciones en su charla y $n$ el número de personas que la entienden, resulta que


\begin{equation}
n = \gamma \left(\frac{1}{2}\right)^q
\end{equation}

donde $\gamma$ es una constante de proporcionalidad.

\item Apunta solo algunos puntos de conclusión. Las personas no pueden recordar más de un par de cosas de una charla, especialmente si escuchan muchas charlas en reuniones grandes.

\item Dirígete a la audiencia, no a la pantalla. Uno de los problemas más comunes que veo es que el conferenciante habla a la pantalla del proyector.

\item Evita hacer sonidos que distraigan. Trata de evitar "Ummm" o "Ahhh" entre oraciones.

\item Pule tus gráficos. Aquí hay una lista de sugerencias para mejores gráficos:


\begin{itemize}

\item Utiliza letras grandes.

\item Mantén los gráficos simples. No muestres gráficos que no necesitarás.

\item Usa color.

\end{itemize}


\item Sé amable al responder preguntas.

\item Utiliza el humor si es posible. Siempre me sorprende ver cómo incluso de una broma realmente tonta se ríe mucho en una charla científica.

\end{enumerate}

\end{document}