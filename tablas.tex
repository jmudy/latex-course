\documentclass{article}

\usepackage{multirow}
\usepackage[spanish]{babel}

\begin{document}

% EJEMPLO 1

\begin{table}
\caption{Precios}
\centering
\label{tab:precios}
\begin{tabular}{|l|l|r|}
\hline
\multicolumn{2}{|c|}{Item}          &            \\ \hline
Animal                & Description & Price (\$) \\ \hline
\multirow{2}{*}{Gnat} & per gram    & 13.65      \\ \cline{2-3} 
                      & each        & 0.01       \\ \hline
Gnu                   & stuffed     & 92.50      \\ \hline
Emu                   & stuffed     & 33.33      \\ \hline
Armadillo             & frozen      & 8.99       \\ \hline
\end{tabular}
\end{table}


% EJEMPLO 2

\begin{table}
\centering
\caption{Horario}
\label{tab:horario}
\begin{tabular}{|c|c|c|c|c|c|}
\hline
            & Lunes  & Martes & Miércoles & Jueves & Viernes \\ \hline
9-11        & A1     & A2     & A1        & A3     & A2      \\ \hline
11-11:30    & \multicolumn{5}{c|}{DESCANSO}                  \\ \hline
11:30-13:30 & A4     & A1     & A3        & A5     & A4      \\ \hline
13:30-14:30 & A2     & A3     & A4        & A1     & A5      \\ \hline
14:30-15    & \multicolumn{5}{c|}{ALMUERZO Y DESCANSO}       \\ \hline
15-18       & \multicolumn{5}{c|}{ESTUDIO}                   \\ \hline
18-20       & CARDIO & TRX    & CARDIO    & TRX    & CARDIO  \\ \hline
\end{tabular}
\end{table}


% EJEMPLO 3

\begin{table}
\centering
\begin{tabular}{lllllllllllllll}
 &  &  &  &  &  &  & 1 &  &  &  &  &  &  &  \\
 &  &  &  &  &  & 1 &  & 1 &  &  &  &  &  &  \\
 &  &  &  &  & 1 &  & 2 &  & 1 &  &  &  &  &  \\
 &  &  &  & 1 &  & 3 &  & 3 &  & 1 &  &  &  &  \\
 &  &  & 1 &  & 4 &  & 6 &  & 4 &  & 1 &  &  &  \\
 &  & 1 &  & 5 &  & 10 &  & 10 &  & 5 &  & 1 &  &  \\
 & 1 &  & 6 &  & 15 &  & 20 &  & 15 &  & 6 &  & 1 &  \\
1 &  & 7 &  & 21 &  & 35 &  & 35 &  & 21 &  & 7 &  & 1
\end{tabular}
\caption{Triángulo de Pascal}
\label{tab:pascal}
\end{table}


% EJEMPLO 4

\begin{table}
\centering
\begin{tabular}{|l|r|r|}
\hline
\multicolumn{1}{|c|}{\textbf{Concepto}} & \multicolumn{1}{c|}{\textbf{Coste}} & \multicolumn{1}{c|}{\textbf{Coste acumulado}} \\ \hline
Picos &  &  \\ \cline{3-3} 
Palas & 100000 &  \\ \cline{3-3} 
Azadones &  & 100000 \\ \hline
Frailes &  &  \\ \cline{3-3} 
Monjas & 150 &  \\ \cline{3-3} 
Pobres &  & 100150 \\ \hline
Guantes & 100 & 100250 \\ \hline
Campanas & 160 & 100410 \\ \hline
Pequeñeces & 100000 & 200410 \\ \hline
\end{tabular}
\caption{Cuentas del Gran Capitán (en miles de escudos)}
\label{tab:gcap}
\end{table}


\end{document}

