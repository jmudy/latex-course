\documentclass{beamer}

% Elegir tema
\usetheme{Berlin}

% Packages
\usepackage[spanish]{babel}

% Commands
\newcommand{\bftt}[1]{\textbf{\texttt{#1}}}
\newcommand{\cmd}[1]{{\color[HTML]{008000}\bftt{#1}}}
\newcommand{\bs}{\char`\\}
\newcommand{\cmdbs}[1]{\cmd{\bs#1}}

\title{Mi primera presentación en Beamer}
\author{Jesús Mudarra Luján}
\institute{Curso de Introducción a \LaTeX}
\date{\today}



\begin{document}

\begin{frame}
\titlepage % el \maketitle de beamer
\end{frame}

\begin{frame}
\tableofcontents
\end{frame}

\section{Primera sección}

\begin{frame}[fragile]
\frametitle{Presentaciones con beamer: Diapositivas}

\begin{itemize}
\item Usa el comando \cmdbs{frametitle} para titular la diapositiva.
\item Después, añade el contenido
\item En fichero fuente de esta diapositiva tiene este aspecto
\end{itemize}

\end{frame}

\section{Segunda sección}

\begin{frame}
\frametitle{Presentaciones con beamer: Introduciendo columnas}

\begin{columns}

% INTRODUCIMOS LA PRIMERA COLUMNA

\begin{column}{0.4\textwidth}
\begin{itemize}
\item Usa \dots
\item El argumento \dots
\item También puedes \dots
\end{itemize}
\end{column}

% INTRODUCIMOS LA SEGUNDA COLUMNA

\begin{column}{0.6\textwidth}
\begin{itemize}
\pause % Para no hacer aparecer toda la diapositiva
\item Aquí \dots
\item Va la segunda \dots
\item columna que hemos creado \dots
\end{itemize}
\end{column}

\end{columns}

\end{frame}

\begin{frame}
\frametitle{Presentaciones con beamer: Destacados}

\begin{itemize}
\item Usar \cmdbs{emph} o \cmdbs{alert} para destacar\\
He de \emph{resaltar} que este punto es \alert{importante}
\item Para especificar en negrita o cursiva respectivamente utilizar \cmdbs{textbf} o \cmdbs{textit}\\
Texto en \textbf{negrita}. Texto en \textit{cursiva}
\item Para especificar un color utilizar el comando \cmdbs{textcolor}\\
\textcolor{red}{Se para} y \textcolor{green}{arranca}.
\end{itemize}

% AÑADIENDO BLOQUES
\begin{block}{Hecho interesante}
Esto es importante.
\end{block}

\begin{alertblock}{Cuento moral}
¡Esto es realmente importante!
\end{alertblock}

\end{frame}

\end{document}
