\documentclass[12pt,a4paper]{article}

\usepackage{amsmath}
\usepackage{natbib}

\begin{document}

\section*{Listas}

\begin{itemize}
\item Picos
\item Palas
\item Azadones
\end{itemize}

\begin{enumerate}
\item Picos
\item Palas
\item Azadones
\end{enumerate}

\begin{equation}
\alpha - \beta = 3
\end{equation}

\subsection*{Uso de comillas}

\begin{verbatim}
Comillas simples: `texto'.
Comillas dobles:  ``texto''.
\end{verbatim}

\subsection*{Algunos caracteres especiales}

\$ \% \& \# !

\section*{Resolviendo la Tarea 1}

In March 2006, Congress raised that ceiling an 
additional \$0.79 trillion to \$8.97 trillion,
which is approximately 68\% of GDP. As of October
4, 2008, the "Emergency Economic Stabilization
Act of 2008" raised the current debt ceiling
to \$11.3 trillion.


\subsection*{Expresiones matemáticas}

Sean $a$ y $b$ enteros positivos distintos, y sea $c = a - b +1$

\subsection*{Otros ejemplos}

$y = c_2 x^2 + c_1 x + c_0$

\subsection*{Diferencias en el uso de llaves para agrupar índices y subíndices}

$F_n = F_n-1 + F_n-2$

$F_n = F_{n-1} + F_{n-2}$

$\mu = A e^{Q/RT}$

$\Omega=\sum_{k=1}^{n}\omega_k$

\subsection*{Para fórmulas grandes que se necesitan mostrar en una sola línea}

\begin{equation}
x = \frac{-b \pm \sqrt{b^2 - 4ac}}{2a}
\end{equation}
donde $a$, $b$ y $c$ son \ldots


\section*{Uso de paquetes}

\subsection*{Ejemplos con AMSMATH}

\begin{equation*}
min_{x,y}{(1-x)^2 + 100(y-x^2)^2} %% Mal!
\end{equation*}

\begin{equation*}
\min_{x,y}{(1-x)^2 + 100(y-x^2)^2} %% Bien!
\end{equation*}

\begin{equation*}
\beta_i=
\frac{\operatorname{Cov}(R_i, R_m)}
	 {\operatorname{Var}(R_m)}
\end{equation*}

\begin{align*}
(x+1)^3 &= (x+1)(x+1)(x+1)\\
		&= (x+1)(x^2 + 2x + 1)\\
		&= x^3 + 3x^2 + 3x + 1
\end{align*}


\section*{Resolviendo la Tarea 2}

Sea $X_1, X_2, \ldots, X_n$ una sucesión de variables aleatorias independientes e idénticamente distribuidas con $\operatorname{E}[X_i] = \mu$ y $\operatorname{Var}[X_i] = \sigma^2  < \infty$, y sea

\begin{equation*}
S_n = \frac{1}{n} \sum_{i}^{n}X_i
\end{equation*}

su media. Entonces, cuando $n$ tiende a infinito, la raíz cuadrada de las variables aleatorias $\sqrt{n} (S_n - \mu)$ convergen a una distribución normal $N(0, \sigma^2)$.

\section*{Introduciendo tablas}

\begin{table}[h!]
\centering
\begin{tabular}{|l|r|r|} \hline
Item   & Qty & Unit \$ \\\hline
Widget &   1 & 199.99 \\
Gadget &   2 & 399.99 \\
Cable  &   3 & 19.99 \\\hline
\end{tabular}
\end{table}

\section*{Añadiendo citas de la bibliografía}

\citet{Brooks1997Methodology}
demuestra que \ldots

Evidentemente, todos los números impares son primos \citep{Jacobson1999Towards}.

Dos formas de citar referencias serían:
\citep{Smith1990Enabling} o \citet{Smith1990Enabling}

\bibliographystyle{plainnat}
\bibliography{references}

\end{document}